% !TEX root=../MA119-Main.tex

\paragraph*{Horizontal and Vertical Lines}

	A \dfn{horizontal line} is defined by an equation $y=b$. The slope of a horizontal line is simply zero.
	A \dfn{vertical line} is defined by an equation $x=a$. The slope of a vertical line is \textbf{undefined}.

	A vertical line gives an example that a graph is not a function of $x$. Indeed,
	the vertical line test fails for a vertical line.



\paragraph*{Explicit Function}
	% \begin{multicols}{2}
		When studying functions, we prefer a clearly expressed function rule. For example, in $f(x)=-\frac23x+1$, the expression $-\frac23x+1$ clearly tells us how to produce outputs.  For a function $f$ defined by an equation, for instance, $2x+3y=3$, to find the function rule (that is an expression), we simply solve the given equation for $y$.
		\[
			\begin{split}
				2x+3y&=3\\
				3y&=-2x+3\\
				y&=-\frac23x+1. \vspace*{-1em}
			\end{split}
		\]
		Now, we get $f(x)=-\frac23x+1$.


\paragraph*{Perpendicular and Parallel Lines}
		Any two vertical lines are parallel. Two non-vertical lines are \dfn{parallel} if and only if they \textbf{have the same slope}.

		A line that is parallel to the line $y=mx+a$ has an equation $y=mx+b$, where $a\neq b$.

		Horizontal lines are perpendicular to vertical lines. Two non-vertical lines are \dfn{perpendicular} if and only if \textbf{the product of their slopes is $-1$}.

		A line that is perpendicular to the line $y=mx+a$ has an equation $y=-\frac1m x+b$.


\paragraph*{Finding Equations for Perpendicular or Parallel Lines}
\mbox{}

		\begin{example}
			Find an equation of the line that is parallel to the line $4x+2y=1$ and passes through the point $(-3, 1)$.
		\end{example}
		
		\begin{solution}
			\begin{enumerate}[label={\textbf{\textup{Step \arabic*.}}~}, itemsep=0em]
				\item Find the slope $m$ of the original line from the slope-intercept form equation by solving for $y$.  $y=-2x+\frac12$. So $m=-2$.
				\item Find the slope $m_\parll$ of the parallel line.\\ \centerline{$m_\parll=m=-2$.}
				\item Use the point-slope form.
				      \[
					      \begin{split}
						      y-1&=-2(x+3)\\
						      y&=-2x-5.
					      \end{split}
				      \]
			\end{enumerate}

		\end{solution}

		\begin{example}
			Find an equation of the line that is perpendicular to the line $4x-2y=1$ and passes through the point $(-2,3)$.
		\end{example}
		
		\begin{solution}
			\begin{enumerate}[label={\textbf{\textup{Step \arabic*.}}~}, itemsep=0em]
				\item Find the slope $m$ of the original line from the slope-intercept form equation by solving for $y$.  $y=2x-\frac12$. So $m=2$.
				\item Find the slope $m_\perp$ of the perpendicular line.\\ \centerline{$m_\perp=-\frac1m=-\frac12$.}
				\item Use the point-slope form.
				      \[
					      \begin{split}
						      y-3&=-\frac12(x+2)\\
						      y&=-\frac{1}{2}x+2.
					      \end{split}
				      \]
			\end{enumerate}

		\end{solution}


\newpage


\begin{exercise}
	Find an equation for each of the following two lines which pass through the same point $(-1, 2)$.\\
	\begin{enumerate*}[label={(\arabic*)~}]
	\item The vertical line.
	\item The horizontal line.\hfill\null
	\end{enumerate*}
\end{exercise}

\vfill

\hfill
\begin{center}
	\raisebox{0.5em}{
		\rotatebox{\rotationdegree}{
			\parbox{\textwidth}{
				\begin{enumerate*}[label={\theexer~(\arabic*)~}]
					\item $x=-1$
					\item $y=2$ \hfill\null
				\end{enumerate*}
			}
		}
	}
\end{center}

\begin{exercise} Line $L$ is defined by the equation $2x-5y=-3$.
	What is the slope $m_\parallel$ of the line that is parallel to the line $L$? 
	What is the slope $m_\perp$ of the line that is perpendicular to the line $L$.
\end{exercise}

\vspace{\stretch{2}}
\hfill
\begin{center}
	\raisebox{0.5em}{
		\rotatebox{\rotationdegree}{
			\parbox{\textwidth}{
				\begin{enumerate*}[label={\theexer~}]
					\item  $m_\parallel=\frac25$ \qquad $m_\perp=-\frac52$  \hfill\null
				\end{enumerate*}
			}
		}
	}
\end{center}

\begin{exercise}
	Line $L_1$ is defined by $3y+5x=7$. Line $L_2$ passes through $(-1, -3)$ and $(4, -8)$. Determine whether $L_1$ and $L_2$ are parallel, perpendicular or neither.
\end{exercise}

\vspace{\stretch{2}}\hfill
\begin{center}
	\raisebox{0.5em}{
		\rotatebox{\rotationdegree}{
			\parbox{\textwidth}{
				\begin{enumerate*}[label={\theexer~}]
					\item neither \hfill\null
				\end{enumerate*}
			}
		}
	}
\end{center}

\newpage



\begin{exercise}
	Find the point-slope form and then the slope-intercept form equations of the line parallel to $3x-y=4$ and passing through the point $(2,-3)$.
\end{exercise}

\vfill\hfill
\begin{center}
	\raisebox{0.5em}{
		\rotatebox{\rotationdegree}{
			\parbox{\textwidth}{
				\begin{enumerate*}[label={\theexer~}]
					\item  $y=3(x-2)-3$ \qquad $y=3x-9$ \hfill\null
				\end{enumerate*}
			}
		}
	}
\end{center}

\begin{exercise}
	Find the slope-intercept form equation of the line that is perpendicular to $4y-2x+3=0$ and passing through the point $(2, -5)$
\end{exercise}

\vfill\hfill
\begin{center}
	\raisebox{0.5em}{
		\rotatebox{\rotationdegree}{
			\parbox{\textwidth}{
				\begin{enumerate*}[label={\theexer~}]
					\item $y=-2x-1$\hfill\null
				\end{enumerate*}
			}
		}
	}
\end{center}

\begin{exercise}
	The line $L_1$ is defined $Ax+By=3$. The line $L_2$ is defined by the equation $Ax+By=2$. The line $L_3$ is defined by $Bx-Ay=1$. Determine whether $L_1$, $L_2$ and $L_3$ are parallel or perpendicular to each other.
\end{exercise}

\vfill\hfill
\begin{center}
	\raisebox{0.5em}{
		\rotatebox{\rotationdegree}{
			\parbox{\textwidth}{
				\begin{enumerate*}[label={\theexer~}]
					\item $L_1\parallel L_2\perp L_3$.
					\hfill\null
				\end{enumerate*}
			}
		}
	}
\end{center}

\newpage 

\begin{exercise} Use the graph of the line $L$ to answer the following questions
\begin{multicols}{2}
	\begin{enumerate}[label={\textup{(\arabic*)~}}]
		\item Find an equation for the line $L$.
		\item Find an equation for the line $L_\perp$ perpendicular to $L$ and passing through $(1,1)$.
		\item Find an equation for the line $L_\parallel$ parallel to $L$ and passing through $(-2,-1)$.
	\end{enumerate}

\columnbreak

\begin{center}
\begin{tikzpicture}[scale=1, every node/.style={scale=0.7}]
\begin{axis}[
	grid=both,
	unit vector ratio=1 1 1,
	ymin=-4,
	ymax=4,
	xmax=4,
	xmin=-4,
	xtick={-10,-9,...,9,10},
	ytick={-10,-9,...,9,10},
	minor tick num=1,
	axis lines = middle,
	xlabel=$x$,
	ylabel=$y$,
	x tick label style={yshift=0.5ex,font={\small}},
	y tick label style={xshift=0.25ex, font={\small}},
	label style ={at={(ticklabel cs:1.1)},
	font={\small}}
]
\addplot[thick, samples=100,domain=-3:3, name path=A, stealth-stealth]   {-1/2*x+1};
 \end{axis}
\end{tikzpicture}
\end{center}

\end{multicols}

\end{exercise}

\vfill\hfill
\begin{center}
	\raisebox{0.5em}{
		\rotatebox{\rotationdegree}{
			\parbox{\textwidth}{
				\begin{enumerate*}[label={\theexer~(\arabic*)~}]
					\item $y=-\frac12x+1$.
					\item $y=2x-1$
					\item $y=-\frac12x-2$
					\hfill\null
				\end{enumerate*}
			}
		}
	}
\end{center}

\begin{exercise} Determine whether the points $(-3,1)$, $(-2,6)$, $(3,5)$ and $(2, 0)$ form a square. Please explain your conclusion.
\end{exercise}

\vspace{\stretch{1.5}}\hfill
\begin{center}
	\raisebox{0.5em}{
		\rotatebox{\rotationdegree}{
			\parbox{\textwidth}{
				\begin{enumerate*}[label={\theexer~}]
					\item The points form a square. Because the rise and the run for each side up to a sign and a switch are the same.
					\hfill\null
				\end{enumerate*}
			}
		}
	}
\end{center}
	