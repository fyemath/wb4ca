% !TEX root=../MA119-Main.tex

\paragraph*{Radical Expressions}
	If $b^2=a$, then we say that $b$ is a \dfn{square root} of $a$. We denote the positive square root of $a$ as $\sqrt{a}$, called the \dfn{principal square root}.

	We call the function $f(x)=\sqrt{x}$ a \dfn{square root function}. As a real-valued function, the domain of the square root function consists of all real numbers $x$ such that $x\ge 0$, in interval notation, the domain is $[0,+\infty)$.

	For any real number $a$, the expression $\sqrt{a^2}$ can be simplified as
	\[
		\sqrt{a^2}=|a|.
	\]

	If $b^3=a$, then we say that $b$ is a \dfn{cube root} of $a$. The cube root of a real number $a$ is denoted by $\sqrt[3]{a}$.

	For any real number $a$, the expression $\sqrt[3]{a^3}$ can be simplified as
	\[
		\sqrt[3]{a^3}=a.
	\]

	In general, if $b^n=a$, then we say that $b$ is an \dfn{$n$-th root} of $a$.  If $n$ is \textbf{even}, the \textbf{positive} $n$-th root of $a$, called the \dfn{principal $n$-th root}, is denoted by $\sqrt[n]{a}$. If $n$ is odd,  the $n$-the root $\sqrt[n]{a}$ of $a$ has the same sign with $a$.

	In $\sqrt[n]{a}$, the symbol $\sqrt{\phantom{a}}$ is called the \dfn{radical sign}, $a$ is called  the \dfn{radicand}, and $n$ is called the \dfn{index}.

	If $n$ is even, then the $n$-th root of a negative number is not a real number.

	For any real number $a$, the expression $\sqrt[n]{a^n}$ can be simplified as
	\begin{enumerate}
		\item  $\sqrt[n]{a^n}=|a|$ if $n$ is even.
		\item  $\sqrt[n]{a^n}=a$ if $n$ is odd.
	\end{enumerate}

	A radical is simplified if the radicand has no perfect power factors against the radical.

	\begin{example}
		Simplify the radical expression using the definition.\\
		\begin{enumerate*}[label={(\arabic*)~}]
			\item $\sqrt{4(y-1)^2}$
			\item $\sqrt[3]{-8x^3y^6}$ \hfill\null
		\end{enumerate*}
	\end{example}
	\begin{solution}\mbox{}
		\begin{enumerate*}[label={\emph{(\arabic*)}~}]
			\item $\sqrt{4(y-1)^2}=\sqrt{[2(y-1)]^2}=2|y-1|$.
			\item $\sqrt[3]{-8x^3y^6}=\sqrt[3]{(-2xy^2)^2}=-2xy^2$. \hfill\null
		\end{enumerate*}
	\end{solution}



\paragraph*{Rational Exponents}
	If $\sqrt[n]{a}$ is a real number, then we define $a^{\frac mn}$ as
	\[
		a^{\frac mn}=\sqrt[n]{a^m}=(\sqrt[n]{a})^m.
	\]

	Rational exponents have the same properties as integral exponents:

	\begin{enumerate*}[label=\arabic*.,afterlabel={\quad}]
		\item \parbox{0.25\textwidth}{$a^m\cdot a^n=a^{m+n}$}
		\item \parbox{0.25\textwidth}{$\dfrac{a^m}{a^n}=a^{m-n}$}
		\item \parbox{0.25\textwidth}{$a^{-\frac mn}=\dfrac{~1~}{~a^{\frac mn}~}$}
	\end{enumerate*}

	\begin{enumerate*}[resume*]
		\item \parbox{0.25\textwidth}{$(a^m)^n=a^{mn}$}
		\item \parbox{0.25\textwidth}{$(ab)^m=a^m\cdot b^m$}
		\item \parbox{0.25\textwidth}{$\left(\dfrac ab\right)^m=\dfrac{a^m}{b^m}$}
	\end{enumerate*}

	\begin{example}
		Simplify the radical expression or the expression with rational exponents. \textbf{Write in radical notation.}\\
		\begin{enumerate*}[label={(\arabic*)~}]
			\item $\sqrt{x}\sqrt[3]{x^2}$
			\item $\sqrt[3]{\sqrt{x^3}}$
			\item $\left(\frac{x^{\frac12}}{x^{-\frac56}}\right )^{\frac14}$
			\item $\sqrt{\frac{x^{-\frac12}y^2}{x^{\frac32}}}$
			\hfill\null
		\end{enumerate*}
	\end{example}
	\begin{solution}\mbox{}
		\begin{enumerate}[label={\emph{(\arabic*)}~}]
			\item
			      \[
				      \sqrt{x}\sqrt[3]{x^2}=x^{\frac12}x^{\frac{2}{3}}=x^{\frac{1}{2}+\frac{2}{3}}=x^\frac{7}{6}=x\sqrt[6]{x}.
			      \]
			\item
			      \[
				      \sqrt[3]{\sqrt{x^3}}=(\sqrt{x^3})^\frac{1}{3}=[(x^3)^{\frac{1}{2}}]^{\frac{1}{3}}=x^{3\cdot\frac{1}{2}\cdot\frac{1}{3}}=x^{\frac{1}{2}}=\sqrt{x}.
			      \]
			\item
			      \[
				      \left(\frac{x^{\frac12}}{x^{-\frac56}}\right)^{\frac14}=(x^{\frac12}x^{\frac56})^{\frac14}=(x^{\frac{1}{2}+\frac{5}{6}})^{\frac{1}{4}}=(x^\frac{4}{3})^{\frac{1}{4}}=x^{\frac{1}{3}}=\sqrt[3]{x}.
				  \]
			\item 
			\[
				\sqrt{\frac{x^{-\frac12}y^2}{x^{\frac32}}}=\sqrt{\frac{y^2}{x^2}}=\sqrt{\left(\frac yx\right)^2}=\left|\frac yx\right|.
			\]
		\end{enumerate}
	\end{solution}

	% \begin{note}
	% 	In general, rewriting radical in rational exponents helps simplify calculations.
	% \end{note}
\newpage


\begin{exercise}
	Evaluate the square root. If the square root is not a real number, state so.\\
	\begin{enumerate*}[label={(\arabic*)~}]
		% \item $-\sqrt{16}$%\hspace{1in}
		\item $-\sqrt{\frac{4}{25}}$%\hspace{1in}
		\item $\sqrt{49}-\sqrt{9}$ %\hspace{1in}
		\item $-\sqrt{-1}$\hfill\null\null
	\end{enumerate*}
\end{exercise}



%%%%%%
\vfill
\begin{center} \hfill
	\raisebox{0.4em}{
		\rotatebox{\rotationdegree}{
			\parbox{\textwidth}{
				\begin{enumerate*}[label={\theexer~(\arabic*)~~}]
					% \item $-4$
					\item $-\frac25$
					\item $4$
					\item not a real number
					\hfill\null
					%\item $(x-2)(x-3)$
				\end{enumerate*}
			}
		}
	}
\end{center}


%\begin{exercise}Find the domain and the indicated function value.
%\begin{multicols}{2}
%\begin{enumerate}[label={\em(\arabic*)},itemsep=0.5in, afterlabel={\quad}]
%\item $f(x)=\sqrt{2x+6}$, \quad $f(-1)$, \quad $f(5)$.
%\end{enumerate}
%
%\begin{enumerate}[resume*]
%\item $f(x)=-\sqrt{9-3x}$, \quad $f(3)$, \quad $f(-\frac{7}{3})$.
%\end{enumerate}
%\end{multicols}
%\end{exercise}

%
%%%%%%%
%\vfill
%\begin{center} \hfill
%\raisebox{0.4em}{\rotatebox{\rotationdegree}{\parbox{\textwidth}{
%	\begin{enumerate*}[label={(\arabic*)~~},afterlabel={\quad}, itemjoin={\hspace{\fill}}]
%	\item $x\geq -3$, $f(-1)=2$, $f(5)=4$
%	\item $x\leq 3$, $f(3)=0$, $f(-\frac73)=-4$.
%	\hfill\null
%	%\item $(x-2)(x-3)$
%	\end{enumerate*}
%}
%}
%}
%\end{center}


\begin{exercise}Simplify the radical expression.\\
	% \vspace*{-0.5\baselineskip}
	% \begin{multicols}{4}
	\begin{enumerate*}[label={(\arabic*)~}]
		\item $\sqrt{(-7x^2)^2}$
		\item $\sqrt{(x+2)^2}$
		\item $\sqrt{25x^2y^6}$
		% \item $\sqrt[3]{-27x^3}$
		% \item $\sqrt[4]{16x^8}$
		% \item $\sqrt[5]{(2x-1)^5}$
		% \item $\sqrt[6]{(-2)^6}$
		% \item $\sqrt[7]{-1}$
		\hspace*{\fill}\null
	\end{enumerate*}
	% \end{multicols}
\end{exercise}


%%%%%%
\vfill
\begin{center} \hfill
	\raisebox{0.4em}{
		\rotatebox{\rotationdegree}{
			\parbox{\textwidth}{
				\begin{enumerate*}[label={\theexer~(\arabic*)~}]
					\item $7x^2$
					\item $|x+2|$
					\item $|5xy^3|$
					% \item $-3x$
					% \item $2x^2$
					% \item $2x-1$
					% \item $2$
					% \item $-1$
					\hfill\null
					%\item $(x-2)(x-3)$
				\end{enumerate*}
			}
		}
	}
\end{center}


\begin{exercise}Simplify the radical expression.\\
	% \vspace*{-0.5\baselineskip}
	% \begin{multicols}{4}
	\begin{enumerate*}[label={(\arabic*)~}]
		% \item $\sqrt{(-7)^2}$
		% \item $\sqrt{(x+2)^2}$
		% \item $\sqrt{25x^2y^6}$
		\item $\sqrt[3]{-27x^3}$
		\item $\sqrt[4]{16x^8}$
		\item $\sqrt[5]{(2x-1)^5}$
		% \item $\sqrt[6]{(-2^6)(-y^{12})}$
		% \item $\sqrt[7]{-1}$
		\hspace*{\fill}\null
	\end{enumerate*}
	% \end{multicols}
\end{exercise}


%%%%%%
\vfill
\begin{center} \hfill
	\raisebox{0.4em}{
		\rotatebox{\rotationdegree}{
			\parbox{\textwidth}{
				\begin{enumerate*}[label={\theexer~(\arabic*)~}]
					% \item $7$
					% \item $|x+2|$
					% \item $|5xy^3|$
					\item $-3x$
					\item $2x^2$
					\item $2x-1$
					% \item $2$
					% \item $-1$
					\hfill\null
					%\item $(x-2)(x-3)$
				\end{enumerate*}
			}
		}
	}
\end{center}

\newpage

\begin{exercise}Simplify the radical expression. Assume all variables are positive. \\
	\begin{enumerate*}[label={(\arabic*)~}]
		\item \parbox{0.25\textwidth}{$\sqrt{50}$}
		\item \parbox{0.25\textwidth}{$\sqrt[3]{-8x^2y^3}$}
		\item \parbox{0.25\textwidth}{$\sqrt[5]{32x^{12}y^2z^8}$}\hfill\null
	\end{enumerate*}
\end{exercise}

%%%%%%
\vfill
\begin{center} \hfill
	\raisebox{0.4em}{
		\rotatebox{\rotationdegree}{
			\parbox{\textwidth}{
				\begin{enumerate*}[label={\theexer~(\arabic*)~}]
					\item \parbox{0.2\textwidth}{ $5\sqrt2$ }
					\item \parbox{0.2\textwidth}{ $-2y\sqrt[3]{x^2}$ }
					\item \parbox{0.2\textwidth}{ $2x^2z\sqrt[5]{x^2y^2z^3}$} \hfill\null
				\end{enumerate*}
			}
		}
	}
\end{center}

\begin{exercise}
	Write the radical expression with rational exponents.
	\\
	\begin{enumerate*}[label={(\arabic*)~}]
		\item $\sqrt[3]{(2x)^5}$
		\item $(\sqrt[5]{3xy})^7$
		\item $\sqrt[4]{(x^2+3)^3}$
		\hfill\null
	\end{enumerate*}
\end{exercise}

%%%%%%
\vfill
\begin{center} \hfill
	\raisebox{0.4em}{
		\rotatebox{\rotationdegree}{
			\parbox{\textwidth}{
				\begin{enumerate*}[label={\theexer~(\arabic*)~~},afterlabel={\quad}, itemjoin={\hspace{\fill}}]
					\item $(2x)^{\frac53}$
					\item $(3xy)^{\frac75}$
					\item $(x^2+3)^{\frac34}$
					\hfill\null
				\end{enumerate*}
			}
		}
	}
\end{center}


\begin{exercise}Write in radical notation and simplify.
	\\
	\begin{enumerate*}[label={(\arabic*)~}]
		\item $4^{\frac32}$
		\item $-81^{\frac 34}$
		\item $\left(\frac{27}{8}\right)^{-\frac{2}{3}}$
		\hfill\null
	\end{enumerate*}
\end{exercise}
%%%%%%
\vfill
\begin{center} \hfill
	\raisebox{0.4em}{
		\rotatebox{\rotationdegree}{
			\parbox{\textwidth}{
				\begin{enumerate*}[label={\theexer~(\arabic*)~}]
					\item $8$
					\item $-27$
					\item $\frac{4}{9}$
					\hfill\null
					%\item $(x-2)(x-3)$
				\end{enumerate*}
			}
		}
	}
\end{center}

\newpage
\begin{exercise}Simplify the expression. Write with radical notations. Assume all variables represent nonnegative numbers.
	\\
	\begin{enumerate*}[label={(\arabic*)~}]
		\item $\dfrac{12x^{\frac12}}{4x^{\frac23}}$
		\item $(x^{-\frac35}y^{\frac12})^{\frac13}$
		\item $\left(\frac{x^{\frac12}}{x^{-\frac13}}\right)^4$
		\hspace{\fill}\null
	\end{enumerate*}
\end{exercise}
%%%%%%
\vfill
\begin{center} \hfill
	\raisebox{0.4em}{
		\rotatebox{\rotationdegree}{
			\parbox{\textwidth}{
				\begin{enumerate*}[label={\theexer~(\arabic*)~}]
					\item $\dfrac3{\sqrt[6]x}$
					\item $\dfrac{\sqrt[6]{y}}{\sqrt[5]{x}}$
					\item $\sqrt[3]{x^{10}}$.
					\hfill\null
				\end{enumerate*}
			}
		}
	}
\end{center}


\begin{exercise}Simplify the expression. Write  in radical notation.  Assume $x$ is nonnegative. \\
	\begin{enumerate*}[label={(\arabic*)~}]
		% \item $\sqrt[4]{16a^{12}y^2}$
		\item $\dfrac{\sqrt{x}}{\sqrt[3]{x}}$
		\item $\sqrt{\sqrt[3]{x}}$
		\item $\sqrt{x}\sqrt[3]{x}$\hfill\null
	\end{enumerate*}
\end{exercise}

%%%%%%
\vfill
\begin{center} \hfill
	\raisebox{0.4em}{
		\rotatebox{\rotationdegree}{
			\parbox{\textwidth}{
				\begin{enumerate*}[label={\theexer~(\arabic*)~}]
					% \item $2a^{3}\sqrt{y}$
					\item $\sqrt[6]{x}$
					\item $\sqrt[6]{x}$
					\item $\sqrt[6]{x^5}$
					\hfill\null
				\end{enumerate*}
			}
		}
	}
\end{center}

\newpage

\begin{exercise}Simplify the expression. Write in radical notation. Assume $x$ is nonnegative. \\
	\begin{enumerate*}[label={(\arabic*)~}]
		\item $\sqrt[5]{32x^{\frac13}}$
		\item $\left(\dfrac{\sqrt[4]{9x}}{3}\right)^{-2}$
		\item $\sqrt{\frac{1}{\sqrt[3]{x^{-2}}}}$
		\hfill\null
	\end{enumerate*}
\end{exercise}

%%%%%%
\vfill
\begin{center} \hfill
	\raisebox{0.4em}{
		\rotatebox{\rotationdegree}{
			\parbox{\textwidth}{
				\begin{enumerate*}[label={\theexer~(\arabic*)~}]
					\item $2\sqrt[15]{x}$
					\item $\frac{3}{\sqrt{x}}$
					\item $\sqrt[3]{x}$
					\hfill\null
				\end{enumerate*}
			}
		}
	}
\end{center}

\begin{exercise}Simplify the expression. Write in radical notation. Assume all variables are nonnegative. \\
	\begin{enumerate*}[label={(\arabic*)~}]
		\item $\sqrt[3]{(-x)^{-2}}\sqrt[2]{x^3}$
		\item $\left(\dfrac{8a^{-\frac{5}{2}}b}{a^{\frac12}b^{-5}}\right)^{-\frac23}$
		\item $\left(\dfrac{y^{-\frac{1}{3}}}{\sqrt[3]{x^{2}}}\right)^{-3}$
		\hfill\null
	\end{enumerate*}
\end{exercise}

%%%%%%
\vfill
\begin{center} \hfill
	\raisebox{0.4em}{
		\rotatebox{\rotationdegree}{
			\parbox{\textwidth}{
				\begin{enumerate*}[label={\theexer~(\arabic*)~}]
					\item $\sqrt[6]{x^5}$
					\item $\frac{a^2}{4b^4}$
					\item $x^2y$
					\hfill\null
				\end{enumerate*}
			}
		}
	}
\end{center}
