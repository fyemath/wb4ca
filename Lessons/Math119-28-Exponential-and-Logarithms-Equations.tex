% !TEX root=../MA119-Main.tex

\paragraph*{Solving Exponential and Logarithmic Equations}
To solve an exponential or logarithmic equation, the first step is to rewrite the equation with a single exponentiation or logarithm. Then we can use the equivalent relation between exponentiation and logarithm to rewrite the equation and solve the resulting equation.

% \begin{trick}\textbf{Solve by Reduction.}

% \end{trick}
	% \begin{enumerate}[label={\textbf{\textup{Step \arabic*.}}~}]
	% 	\item  Rewrite the equation in the form $b^u=c$, where $u$ is an algebraic expression.
	% 	\item Take logarithm of both sides with respect to the base $b$ (or using the definition to convert).
	% 	\item Simplify the resulting equation $u=\log_bc$ and solve for the variable.
	% \end{enumerate}
	\begin{example}
		Solve the equation \quad $10^{2x-1}-5=0$.
	\end{example}
	\begin{solution}\mbox{}
		% \vspace*{-0.5\baselineskip}
		\begin{enumerate}[label={\textbf{\textup{Step \arabic*.}}~}]
			\item  Rewrite the equation in the form $b^u=c$: 
			$$10^{2x-1}=5.$$
			\item Take logarithm of both sides and simplify: $$2x-1=\log 5.$$
			\item Solve the resulting equation: 
			$$x=\frac{1}{2}(\log5+1).$$
		\end{enumerate}
	\end{solution}

	\begin{example}
		Solve the equation\quad $\log_2 x + \log_2 (x - 2) = 3$.
	\end{example}
	\begin{solution}\mbox{}
		% \vspace*{-0.5\baselineskip}
		\begin{enumerate}[label={\textbf{\textup{Step \arabic*.}}~}]
			\item Rewrite the equation in the form $\log_bu=c$:
			      \[\log_2(x(x - 2)) = 3\]
			\item Rewrite the equation in the exponential form (moving the base):
			      \[x(x-2)=2^3\]
			\item Solve the resulting equation $x^2-2x-8=0$. The solutions are $x=-2$ and $x=4$
			\item Check proposed solutions. Both $x$ and $x-2$ has to be positive. So $x=-2$ is not a solution of the original equation.  When $x=4$, we have $\log_24+\log_22=2+1=3$.  So $x=4$ is a solution.
		\end{enumerate}
	\end{solution}

\paragraph*{Solving Compound Interest Model}\mbox{}

	\begin{example}
		A check of \$5000 was deposited in a savings account with an annual interest rate $6\%$ which is compounded monthly. How many years will it take for the money to raise by 20\%?
	\end{example}
	\begin{solution}
		The question tells us the following information:
		$P=5000$, $r=0.06$, $n=12$, and $A=5000\cdot (1+0.2)=6000$. What we want to know is the number of years $t$. The compound interest model tells us that $t$ satisfies the following equation:
		\[
			6000=5000\left(1+\frac{0.06}{12}\right)^{12t}.
		\]
		This is an exponential equation and can be solve using logarithms.
		\[
			\begin{split}
				5000\left(1+\frac{0.06}{12}\right)^{12t}&=6000\\
				\left(1+\frac{0.06}{12}\right)^{12t}&=1.2\\
				12t\cdot \log\left(1+0.06\div 12\right)&=1.2\\
				12t&=\log(1.2)\div \log(1+0.06\div 12)\\
				t&=\log(1.2)\div\log(1+0.06\div 12)\div 12\approx 3.
			\end{split}
		\]
		So it takes about 3 years for the savings to raise by 20\%.
	\end{solution}

	\begin{note}
		When solving exponential and logarithmic equations, you may also use the one-to-one property if both sides are powers with the same base or logarithms with the same base.
	\end{note}
\newpage


\begin{exercise}
	Solve the exponential equation.
	
	\begin{enumerate*}[label={(\arabic*)~}]
		\item $2^{x-1}=4$
		\item $7e^{2x}-5=58$
		\hfill\null
	\end{enumerate*}
\end{exercise}
%%%%%%
\vfill
\begin{center} \hfill
	\raisebox{0.5em}{
		\rotatebox{\rotationdegree}{
			\parbox{\textwidth}{
				\begin{enumerate*}[label={\theexer~(\arabic*)~}]
					\item $x=3$
					\item $x=\ln3$
					\hfill\null
				\end{enumerate*}
			}
		}
	}
\end{center}


\begin{exercise}
	Solve the exponential equation.
	
	\begin{enumerate*}[label={(\arabic*)~}]
		\item $3^{x^2-2x}=e^{-\ln3}$
		\item $2^{(x+1)}=3^{(1-x)}$
		\hfill\null
	\end{enumerate*}
\end{exercise}
%%%%%%
\vfill
\begin{center} \hfill
	\raisebox{0.5em}{
		\rotatebox{\rotationdegree}{
			\parbox{\textwidth}{
				\begin{enumerate*}[label={\theexer~(\arabic*)~}]
					\item $x=1$
					\item $x=\frac{\ln 3-\ln 2}{\ln 6}$
					\hfill\null
				\end{enumerate*}
			}
		}
	}
\end{center}

\newpage

\begin{exercise}
	Solve the logarithmic equation.\\
	\noindent
	\begin{enumerate*}[label={(\arabic*)~}]
		\item $\log_5x+\log_5(4x-1)=1$
		\item $\ln \sqrt{x+1}=1$
		\hfill\null
	\end{enumerate*}
	\vfill
\begin{center} \hfill
	\raisebox{0.5em}{
		\rotatebox{\rotationdegree}{
			\parbox{\textwidth}{
				\begin{enumerate*}[label={\theexer~(\arabic*)~}]
					\item $x=\frac54$
					\item $x=e^2-1$
					\hfill\null
				\end{enumerate*}
			}
		}
	}
\end{center}
\end{exercise}



\begin{exercise}
	Solve the logarithmic equation.\\
	\noindent
	\begin{enumerate*}[label={(\arabic*)~}]
		\item $\log_2(x+2)-\log_2(x-5)=3$
		\item $\log_3(x-5)=2-\log_3(x+3)$
		\hfill\null
	\end{enumerate*}
\end{exercise}

%%%%%%
\vfill
\begin{center} \hfill
	\raisebox{0.5em}{
		\rotatebox{\rotationdegree}{
			\parbox{\textwidth}{
				\begin{enumerate*}[label={\theexer~(\arabic*)~}]
					\item $x=6$
					\item $x=6$
					\hfill\null
				\end{enumerate*}
			}
		}
	}
\end{center}

\newpage

\begin{exercise}
	For the given function, find values of $x$ satisfying the given equation.

	\begin{enumerate*}[label={(\arabic*)~}]
		\item $f(x)=\log_4x-2\log_4(x+1)$,\quad $f(x)=-1$
		\item $g(x)=\log(2-5x)+\log(-x)$,\quad $g(x)=1$
		\hfill\null
	\end{enumerate*}
\end{exercise}

%%%%%%
\vfill
\begin{center} \hfill
	\raisebox{0.5em}{
		\rotatebox{\rotationdegree}{
			\parbox{\textwidth}{
				\begin{enumerate*}[label={\theexer~(\arabic*)~}]
					\item $x=1$
					\item $x=\frac{1-\sqrt{51}}{5}$
					\hfill\null
				\end{enumerate*}
			}
		}
	}
\end{center}


\begin{exercise}
	Find intersections of the given pairs of curves.

	\begin{enumerate*}[label={(\arabic*)~}]
		\item $f(x)=e^{x^2}$ and~ $g(x)=e^x+12$.
		\item $f(x)=\log_7\left(\frac12(x+2)\right)$ and~ $g(x)=1-\log_7(x-3)$
		\hfill\null
	\end{enumerate*}
\end{exercise}
%%%%%%
\vfill
\begin{center} \hfill
	\raisebox{0.5em}{
		\rotatebox{\rotationdegree}{
			\parbox{\textwidth}{
				\begin{enumerate*}[label={\theexer~(\arabic*)~}]
					\item $x=2\ln 2$
					\item $x=5$
					\hfill\null
				\end{enumerate*}
			}
		}
	}
\end{center}

\newpage

\begin{exercise}
	Using the formula $A=P(1+\frac rn)^{nt}$ to determine how many years, to the nearest hundredth,  it will take to double an investment \$20,000 at the interest rate 5\% compounded monthly.
\end{exercise}

%%%%%%
\vfill
\begin{center} \hfill
	\raisebox{0.5em}{
		\rotatebox{\rotationdegree}{
			\parbox{\textwidth}{
				\begin{enumerate*}[label={\theexer~}]
					\item 	$14$ years \hfill\null
				\end{enumerate*}
			}
		}
	}
\end{center}


\begin{exercise}
Newton's Law of Cooling states that the temperature $T$ of an object at any time $t$ satisfying the equation $T=T_s+(T_0-T_s)e^{-rt}$,
where $T_s$ is the the temperature of the surrounding environment,
$T_0$ is the initial temperature of the object,
and $r$ is positive constant characteristic of the system,
which is in units of ${\displaystyle {\text{time}}^{-1}}{\displaystyle {\text{time}}^{-1}}$.
In a room with a temperature of $22 ^\circ C$, a cup of tea of $97 ^\circ C$ was freshly brewed. 
Suppose that $r=\ln 5/20~~\text{minute}^{-1} $. In how many minutes, the temperature of the tea will be $37 ^\circ C$?
\end{exercise}

%%%%%%
\vfill
\begin{center} \hfill
	\raisebox{0.5em}{
		\rotatebox{\rotationdegree}{
			\parbox{\textwidth}{
				\begin{enumerate*}[label={\theexer~}]
					\item 	$20$ minutes \hfill\null
				\end{enumerate*}
			}
		}
	}
\end{center}
