% !TEX root = ../MA119-Main.tex

\paragraph*{Properties of Exponents}
For an integer $n$, and an expression $x$, the mathematical operation of the $n$-times repeated multiplication of $x$  is call exponentiation, written as $x^n$, that is,
	\[
		% x^n=\underset{n \text{~factors of~}x}{\underbrace{x\cdot x \cdots x}}.
		x^n=\undercbrace{x\cdot x \cdots x}_{n \text{~factors of~}x}.
	\]

	
	In the notation $x^n$, $n$ is called \dfn{the exponent}, $x$ is called \dfn{the base}, and $x^n$ is called \dfn{the power}, read as ``$x$ raised to the $n$-th power",  ``$x$ to the $n$-th power", ``$x$ to the $n$-th", ``$x$ to the power of $n$" or ``$x$ to the $n$".
	%%%%
	% \undercbrace is the command provided by the mtpro2 package see https://tex.stackexchange.com/questions/30090/how-can-i-get-text-underneath-an-underbrace
	%%%

	\begin{enumerate}
		\item
		      \begin{multicols}{2}
			      \textbf{The product rule}
			      \[x^m\cdot x^n=x^{m+n}.\]

			      \textbf{Example:}
			      \[2x^2\cdot (-3x^3)=-6x^5.\]
		      \end{multicols}
		\item
		      \begin{multicols}{2}
			      \textbf{The quotient rule} (for $x\neq 0$.)
			      \[
				      \dfrac{x^m}{x^n}=
				      \begin{cases}
					      x^{m-n}            & \parbox{0.25\textwidth}{if $m$ is greater than \\ or equal to $n$.}\\[1em]
					      \dfrac{1}{x^{n-m}} & \text{if~} m \text{~is less than ~} n.
				      \end{cases}
			      \]

			      \textbf{Example:}
			      \[\frac{15x^5}{5x^2}=3x^3,\quad\quad \frac{-3x^2}{6x^3}=-\frac{1}{2x}\]
		      \end{multicols}
		\item
		      \begin{multicols}{2}
			      \textbf{The zero exponent rule} (for $x\neq 0$.)
			      \[x^0=1.\]

			      \textbf{Example:}
			      \[(-2)^0=1, \quad\quad -x^0=-1\]
		      \end{multicols}
		\item
		      \begin{multicols}{2}
			      \textbf{The negative exponent rule} (for $x\neq 0$.)
			      \[x^{-n}=\dfrac{1}{x^n} \quad\text{and}\quad \dfrac{1}{x^{-n}}=x^n.\]

			      \textbf{Example:}
			      \[(-2)^{-3}=\frac{1}{(-2)^3}=-\frac18, \quad\quad \frac{x^{-2}}{x^{-3}}=\frac{x^3}{x^2}=x.\]
		      \end{multicols}
		\item
		      \begin{multicols}{2}
			      \textbf{The power to a power rule}:
			      \[\left(x^a\right)^b=x^{ab}.\]

			      \textbf{Example:}
			      \[\left(2^{2}\right)^3=2^6=64, \quad\quad \left(x^2\right)^3=x^6.\]
		      \end{multicols}
		\item
		      \begin{multicols}{2}
			      \textbf{The product raised to a power rule}:
				  \[(xy)^n=x^ny^n.\]
				  
				%   \columnbreak
				  
			      \textbf{Example:}
				  \[\left(-2x\right)^{2}=(-2)^2x^2=4x^2\] 
		      \end{multicols}
		\item 
		% \vspace*{-2\baselineskip}	  
		\begin{multicols}{2}
			      \textbf{The quotient raised to a power rule} (for $y\neq 0$.)
			      \[\left(\dfrac{x}{y}\right)^n=\dfrac{x^n}{y^n}.\]

			      \textbf{Example:}
			      \[
					  \left(\dfrac{x}{-2}\right)^{3}=\dfrac{x^3}{(-2)^3}=-\dfrac{x^3}{8}.
					%   ~ \left(\dfrac{-a^2}{b^3}\right)^2=\dfrac{(-a^2)^2}{(b^3)^2}=\dfrac{a^4}{b^6}.
					\]
			  \end{multicols}
			  
	\end{enumerate}



\paragraph*{Order of Basic Mathematical Operations}
	In mathematics, the order of operations reflects conventions about which procedure should be performed first. There are four levels (from the highest to the lowest):\\
	\centerline{\textbf{Parenthesis}; \textbf{Exponentiation}; \textbf{Multiplication and Division}; \textbf{Addition and Subtraction}.} \\
	Within the same level, the convention is to perform from the left to the right.


	\begin{example}
		Simplify. \textbf{Write with positive exponents.} 
		$$
		\left(\dfrac{2y^{-2}z^{-5}}{4x^{-3}y^6}\right)^{-4}.
		$$
	\end{example}
	% \vspace*{-\baselineskip}
	\begin{solution}
		$$
		\left(\dfrac{2y^{-2}z^{-5}}{4x^{-3}y^6}\right)^{-4}
		=\left(\dfrac{x^3}{2z^{5}y^8}\right)^{-4}
		=\left(\dfrac{2z^{5}y^8}{x^3}\right)^4
		=\dfrac{2^4(z^{5})^4(y^8)^4}{(x^3)^4}
		=\dfrac{16y^{32}z^{20}}{x^{12}}.
		$$
	\end{solution}

	\begin{trick} \textbf{Simplify (at least partially) the problem first.}

		To avoid mistakes when working with negative exponents, it's better to apply the negative exponent rule to change negative exponents to positive exponents and simplify the base first.
	\end{trick}

	% \vfill

\newpage
\begin{exercise}
	Simplify. \textbf{Write with positive exponents.}

	\noindent
	\begin{enumerate*}[label={(\arabic*)~}]
		% \item $a^2a^3a^5$
		\item $(3a^2b^3c^2)(4abc^2)(2b^2c^3)$
		\item $\dfrac{4y^3z^0}{x^2y^2}$
		\item $(-2)^{-3}$
		\hfill\null
	\end{enumerate*}
\end{exercise}

\vfill
\begin{center}
	\hfill
	\raisebox{0.4em}{
		\rotatebox{\rotationdegree}{
			\parbox{\textwidth}{
				\begin{enumerate*}[label={\theexer~(\arabic*)~}]
					% \item  $a^{10}$
					\item  $24a^3b^6c^7$
					\item $\frac{4y}{x^2}$
					\item $-\frac{1}{8}$
					\hfill\null
				\end{enumerate*}
			}
		}
	}
\end{center}


\begin{exercise}
	Simplify. \textbf{Write with positive exponents.}

	\noindent
	\begin{enumerate*}[label={(\arabic*)~}]
		\item $\dfrac{-u^0v^{15}}{v^{16}}$
		\item $(-2a^3b^2c^0)^3$
		\item $\dfrac{m^5 n^{2}}{(mn)^3}$
		\hfill\null
	\end{enumerate*}
\end{exercise}

\vfill

\begin{center}
	\hfill
	\raisebox{0.4em}{
		\rotatebox{\rotationdegree}{
			\parbox{\textwidth}{
				\begin{enumerate*}[label={\theexer~(\arabic*)~}]
					\item $-\frac{1}{v}$
					\item $-8a^9b^6$
					\item $\frac{m^2}{n}$
					\hfill\null
				\end{enumerate*}
			}}}
\end{center}

\newpage

\begin{exercise} Simplify. \textbf{Write with positive exponents.}

	\noindent
	\begin{enumerate*}[label={(\arabic*)~}]
		\item $(-3a^2x^3)^{-2}$
		\item $\left(\dfrac{-x^0y^3}{2wz^2}\right)^3$
		\item $\dfrac{3^{-2}a^{-3}b^5}{x^{-3}y^{-4}}$
		\hfill\null
	\end{enumerate*}
\end{exercise}

\vfill
\begin{center}
	\hfill
	\raisebox{0.4em}{
		\rotatebox{\rotationdegree}{
			\parbox{\textwidth}{
				\begin{enumerate*}[label={\theexer~(\arabic*)~}]
					\item $\frac{1}{9a^4x^6}$
					\item $-\frac{y^9}{8w^3z^6}$
					\item  $\frac{b^5x^3y^4}{9a^3}$
					\hfill\null
				\end{enumerate*}
			}
		}
	}
\end{center}


\begin{exercise}
	Simplify. \textbf{Write with positive exponents.}

	\noindent
	\begin{enumerate*}[label={(\arabic*)~}]
		\item $\left(-x^{-1}(-y)^2\right)^3$
		\item $\left(\dfrac{6x^{-2}y^5}{2y^{-3}z^{-11}}\right)^{-3}$
		\item $\dfrac{(3 x^{2} y^{-1})^{-3}(2 x^{-3} y^{2})^{-1}}{(x^{6} y^{-5})^{-2}}$
		\hfill\null
	\end{enumerate*}
\end{exercise}

\vfill
\begin{center}
	\hfill
	\raisebox{0.4em}{
		\rotatebox{\rotationdegree}{
			\parbox{\textwidth}{
				\begin{enumerate*}[label={\theexer~(\arabic*)~}]
					\item  $-\frac{y^6}{x^3}$
					\item $\frac{x^6}{27y^{24}z^{33}}$
					\item $\frac{x^{9}}{54 y^{9}}$
					\hfill\null
				\end{enumerate*}
			}
		}
	}
\end{center}

\newpage

\begin{exercise}
	A pizza shop offer 12-inches pizza and 8-inches pizza at the price \$12 and \$6 respectively.  With \$12, would you like to order one 12-inches and two 8-inches. Please explain.
\end{exercise}

\vfill
\begin{center}
	\hfill
	\raisebox{0.4em}{
		\rotatebox{\rotationdegree}{
			\parbox{\textwidth}{
				\begin{enumerate*}[label={\theexer~}]
					\item 12-inches pizza is better. Because the total area is bigger.
					\hfill\null
				\end{enumerate*}
			}
		}
	}
\end{center}

\begin{exercise}
	A store has large size and small size watermelons. A large one cost \$4 and a small one \$1. The circumference of a larger watermelon is approximately twice of the circumference of a smaller one. Which size of watermelons is cheaper if it was sold by weight? 
\end{exercise}

\vfill
\begin{center}
	\hfill
	\raisebox{0.4em}{
		\rotatebox{\rotationdegree}{
			\parbox{\textwidth}{
				\begin{enumerate*}[label={\theexer~}]
					\item The larger one. Because the volume is 8 times larger.
					\hfill\null
				\end{enumerate*}
			}
		}
	}
\end{center}
