% !TEX root=../MA119-Main.tex

\begin{tcolorbox}[colback=white,colframe=cyan, title filled=false, coltitle=cyan, enhanced, attach boxed title to top center={yshift=-3mm,yshifttext=-1mm}, fonttitle=\bfseries, boxed title style={size=small,colback=white}, before upper={\parindent15pt},
title={Horizontal and vertical lines}]
A \dfn{horizontal line} is defined by an equation $y=b$. The slope of a horizontal line is simply zero.
A \dfn{vertical line} is defined by an equation $x=a$. The slope of a vertical line is \textbf{undefined}. 

A vertical line gives an example that a graph is not a function of $x$. Indeed, 
the vertical line test fails. 



\end{tcolorbox}

\begin{tcolorbox}[colback=white,colframe=cyan, title filled=false, coltitle=cyan, enhanced, attach boxed title to top center={yshift=-3mm,yshifttext=-1mm}, fonttitle=\bfseries, boxed title style={size=small,colback=white}, before upper={\parindent15pt},
title={Get the explicit function from an equation}]
\begin{multicols}{2}
When studying equations of lines, we prefer the equation in the form $y=mx+b$ which tells us the slope and the $y$-intercept.  For a equation in general form, for instance, $2x+3y=3$, to rewrite it into the slope-intercept form, we simply solve for $y$ from the equation.

\columnbreak

\mbox{}
\vspace*{-1em}
\[\begin{split}
2x+3y&=3\\
3y&=-2x+3\\
y&=-\frac23x+1. \vspace*{-1em}
\end{split}\]
Now, we get $f(x)=-\frac23x-1$.
\end{multicols}


\end{tcolorbox}



\begin{tcolorbox}[colback=white,colframe=cyan, title filled=false, coltitle=cyan, enhanced, attach boxed title to top center={yshift=-3mm,yshifttext=-1mm}, fonttitle=\bfseries, boxed title style={size=small,colback=white}, before upper={\parindent15pt},
title={Point-slope form equation of a line}]

When the slope $m$ of a line and a point $(x_0, y_0)$ on the line are given, we can write down an equation for the line in the following form, called the point-slope form:\\
\centerline{$y-y_0=m(x-x_0)$.}\\ 
This equation simply comes from the slope formula\\
\centerline{$m=\dfrac{y-y_0}{x-x_0}$.}\\ 
\end{tcolorbox}



\begin{tcolorbox}[colback=white,colframe=cyan, title filled=false, coltitle=cyan, enhanced, attach boxed title to top center={yshift=-3mm,yshifttext=-1mm}, fonttitle=\bfseries, boxed title style={size=small,colback=white}, before upper={\parindent15pt},
title={Perpendicular and parallel lines}]
\begin{multicols}{2}
Any two vertical lines are parallel. Two non-vertical lines are \dfn{parallel} if and only if they \textbf{have the same slope}. 

A line that is parallel to the line $y=mx+a$ has an equation $y=mx+b$, where $a\neq b$.


\columnbreak

Horizontal lines are perpendicular to vertical lines. Two non-vertical lines are \dfn{perpendicular} if and only if \textbf{the product of their slopes is $-1$}. 

A line that is perpendicular to the line $y=mx+a$ has an equation $y=-\frac1m x+b$.


\end{multicols}
\end{tcolorbox}

\begin{tcolorbox}[colback=white,colframe=cyan, title filled=false, coltitle=cyan, enhanced, attach boxed title to top center={yshift=-3mm,yshifttext=-1mm}, fonttitle=\bfseries, boxed title style={size=small,colback=white}, before upper={\parindent15pt},
title={Find equations for perpendicular or parallel lines}]
\begin{multicols}{2}
\begin{example}
Find an equation of the line that is perpendicular to the line $4x+2y=1$ and pass through the point $(-2,3)$.
\begin{enumerate}[label={Step \arabic*.~~~}, itemsep=0em]
\item Find the slope $m$ of the original line. For that, we can find the slope-intercept form equation by solving for $y$:  $y=2x-\frac12$. So $m=2$.
\item Find the slope $m_\perp$ of the perpendicular line.\\ \centerline{$m_\perp=-\frac1m=-\frac12$.}
\item Use the point-slope form.\\\centerline{$y-3=-\frac12(x+2)$.}
\end{enumerate}

\end{example}

\columnbreak

\begin{example}
Find an equation of the line that is perpendicular to the line $4x-2y=1$ and pass through the point $(-3, 1)$.
\begin{enumerate}[label={Step \arabic*.~~~}, itemsep=0em]
\item Find the slope $m$ of the original line. For that, we can find the slope-intercept form equation by solving for $y$:  $y=-2x+\frac12$. So $m=-2$.
\item Find the slope $m_\parll$ of the perpendicular line.\\ \centerline{$m_\parll=m=-2$.}
\item Use the point-slope form.\\
\centerline{$y-1=-2(x+3)$.}
\end{enumerate}

\end{example}

\end{multicols}
\end{tcolorbox}








\newpage
\begin{exercise}\label{par}
Find the point-slope equation of a line with slope $5$ that passes though $(-2, 1)$.
\end{exercise}

\vfill

\begin{exercise}\label{par}
Find the point-slope equation of a line passing thought $(3, -2)$ and $(1,4)$.
\end{exercise}

\vfill


\begin{exercise}\label{par}
Find the slope of a line that is \begin{enumerate*}[label={\textup{~~(\arabic*)~~}}]\item parallel and \item perpendicular\end{enumerate*} to the line  $2x-5y=-3$.
\end{exercise}

\vfill



\begin{exercise}\label{par}
Find the point-slope form and then the slope-intercept form of the equation of the line parallel to $3x-y=4$ that passes through the point $(2,-3)$.
\end{exercise}

\vfill
\begin{exercise}\label{perp}
Find the slope-intercept form of the equation of the line that is perpendicular to $4y-2x+3=0$ and passes through the point $(2, -5)$
\end{exercise}


\vfill

%\begin{exercise}
%Graph respectively the parallel and perpendicular lines your found in Practice \ref{par} and \ref{perp}.
%\begin{multicols}{2}
%\begin{center}
%\begin{tikzpicture}[scale=1, every node/.style={scale=0.7}]
% \begin{axis}[grid=both, unit vector ratio=1 1 1, ymin=-5,ymax=5,xmax=5,xmin=-5,xtick={-10,-9,...,9,10},ytick={-10,-9,...,9,10},minor tick num=1,axis lines = middle,xlabel=$x$,ylabel=$y$, 
%x tick label style={yshift=0.5ex,font={\small}}, y tick label style={xshift=0.25ex, font={\small}}, label style ={at={(ticklabel cs:1.1)}, font={\small}}
%]
%% \addplot[thick, samples=100,domain=-3:3, name path=A, stealth-stealth]   {-1/2*x+1};           
%%  \node[draw,shape=circle, minimum size=2mm,inner sep=0pt,outer sep=0pt, fill=black] at (0,1) {}; 
%%  \node[draw,shape=circle, minimum size=2mm,inner sep=0pt,outer sep=0pt, fill=black] at (2,0) {};                
%  \end{axis}
%\end{tikzpicture}
%\end{center}
%
%\columnbreak
%
%\begin{center}
%\begin{tikzpicture}[scale=1, every node/.style={scale=0.7}]
% \begin{axis}[grid=both, unit vector ratio=1 1 1, ymin=-5,ymax=5,xmax=5,xmin=-5,xtick={-10,-9,...,9,10},ytick={-10,-9,...,9,10},minor tick num=1,axis lines = middle,xlabel=$x$,ylabel=$y$, 
%x tick label style={yshift=0.5ex,font={\small}}, y tick label style={xshift=0.25ex, font={\small}}, label style ={at={(ticklabel cs:1.1)}, font={\small}}
%]
%% \addplot[thick, samples=100,domain=-3:3, name path=A, stealth-stealth]   {-1/2*x+1};           
%%  \node[draw,shape=circle, minimum size=2mm,inner sep=0pt,outer sep=0pt, fill=black] at (0,1) {};  
%%  \node[draw,shape=circle, minimum size=2mm,inner sep=0pt,outer sep=0pt, fill=black] at (-2,2) {};   
%%  \addplot[blue, draw, -stealth] (0, 1)--(0,2) node[midway, xshift=1.5em] {rise=1}; 
%%  \addplot[red, draw, -stealth] (0, 2)--(-2,2) node[midway, yshift=0.5em] {run=-2};          
%  \end{axis}
%\end{tikzpicture}
%\end{center}
%
%\end{multicols}
%
%\end{exercise}


\vspace{\fill}
\hrule

\begin{center}
\raisebox{0.5em}{
	\rotatebox{\rotationdegree}{
		\parbox{\textwidth}{%\hspace*{5em}

			\begin{enumerate*}[label={(\arabic*)~}]
			\item  $y=5(x+2)+1$.
			\item  $y=-3(x-1)+4$.
			\item  $m_\parallel=\frac25$; $m_\perp=-\frac52$. 
			\item  $y+3=3(x-2)$; $y=3x-9$.
			\item $y=-2x-1$.
			\end{enumerate*}

		}
	}
}
\end{center}


