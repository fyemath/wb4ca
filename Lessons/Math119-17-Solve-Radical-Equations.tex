% !TEX root=../MA119-Main.tex

\paragraph*{Solve Radical Equations by Taking a Power}
	The idea to solve a radical equation $\sqrt[n]{X}=a$ is to first take $n$-th power of both sides to get rid of the radical sign, that is $X=a^n$ and then solve the resulting equation.

	\begin{trick}\textbf{Solve by Reduction.}
		The goal to solve a single variable equation is to isolate the variable. 
		When an equation involves radical expressions, you can not isolate the variable arithmetically without eliminating the radical sign. To remove a radical sign, you make take a power. However, you'd better to isolate it first. Because simply taking powers of both sides may create new radical expressions.
	\end{trick}

	\begin{example}
		Solve the equation 
		$x-\sqrt{x+1}=1.$
	\end{example}
	\begin{solution}
		\begin{multicols}{2}
			\begin{enumerate}[label={\textbf{\textup{Step \arabic*.}}~}]
				\item Arrange terms so that one radical is isolated on one side of the equation.
				      \[x-1=\sqrt{x+1}\]
				\item Square both sides to eliminate the square root.
				      \[(x-1)^2=x+1\]
				\item Solve the resulting equation.
				      %(If it still contains radicals, repeat Step 1. and Step 2.)
				      %\begin{multicols}{2}
				      \[
					      \begin{split}
						      x^2-2x+1&=x+1\\
						      x^2-3x&=0\\
						      x(x-3)&=0
					      \end{split}
				      \]
				      %	\columnbreak
				      \begin{alignat*}{3}
					      x & =0 &  & \ctc{or} & x-3 & =0 \\
					      x & =0 &  & \ctc{or} & x   & =3
				      \end{alignat*}
				      %\end{multicols}
				\item Check all proposed solutions.
				      Plug $x=0$ into the original equation, we see that the left hand side is  $0-\sqrt{0+1}=0-\sqrt{1}=0-1=-1$ which is not equal to the right hand side. So $x=0$ cannot be a solution.

				      Plug $x=3$ into the original equation, we see that the left hand side is $3-\sqrt{3+1}=3-\sqrt{4}=3-2=1$. So $x=3$ is a solution.
			\end{enumerate}
		\end{multicols}
	\end{solution}

	% \bigskip
	% \hrule
	% \bigskip

	\begin{example}
		Solve the equation $\sqrt{x-1}-\sqrt{x-6}=1.$
	\end{example}
	\begin{solution}
		\begin{multicols}{2}
			\begin{enumerate}[label={\textbf{\textup{Step \arabic*.}}~}]
				\item Isolated one radical.
				      \[
					      \sqrt{x-1}=\sqrt{x-6}+1\\
				      \]
				\item Square both sides to remove radical sign and then isolate the remaining radical.
				      \[
					      \begin{split}
						      x-1&=(x-6)+2\sqrt{x-6}+1\\
						      x-1&=x-5+2\sqrt{x-6}\\
						      4&=2\sqrt{x-6}\\
						      2&=\sqrt{x-6}.
					      \end{split}
				      \]
				\item Square both sides to remove the radical sign and then solve.
				      \[
					      \begin{split}
						      \sqrt{x-6}&=2\\
						      x-6&=4\\
						      x&=10.
					      \end{split}
				      \]
				      Since $10-1>0$ and $10-6>0$, $x=10$ is a valid solution. Indeed,
				      \[
					      \sqrt{10-1}-\sqrt{10-6}=\sqrt{9}-\sqrt{4}=3-2=1.
				      \]
			\end{enumerate}
		\end{multicols}
	\end{solution}


	% \bigskip
	% \hrule
	% \bigskip

	\begin{example}
		Solve the equation $-2\sqrt[3]{x-4}=6.$
	\end{example}
	\begin{solution}
		% \begin{multicols}{2}
			\begin{enumerate}[label={\textbf{\textup{Step \arabic*.}}~}]
				\item Isolated the radical.
					  \[
						  \sqrt[3]{x-4}=-3
					  \]
				\item Cube both sides to eliminate the cube root and then solve the resulting equation.
				      \[
					      \begin{split}
						      x-4&=(-3)^3\\
						      x-4&=-27\\
						      x&=-23
					      \end{split}
				      \]
				      The solution is $x=-23$.
			\end{enumerate}
		% \end{multicols}
	\end{solution}

\newpage

\begin{exercise}
	Solve each radical equation.

	\begin{enumerate*}[label={(\arabic*)~}, series=numinpar]
		\item $\sqrt{3x+1}=4$
		\item $\sqrt{2x-1}-5=0$
		\hfill\null
	\end{enumerate*}

\end{exercise}

	\vfill

	\begin{center} \hfill
		\raisebox{0.4em}{
			\rotatebox{\rotationdegree}{
				\parbox{\textwidth}{
					\begin{enumerate*}[label={\theexer~(\arabic*)~}]
						\item $x=5$
						\item $x=13$
						\hfill\null
					\end{enumerate*}
				}
			}
		}
	\end{center}

	
\begin{exercise}
	Solve each radical equation.

	\begin{enumerate*}[label={(\arabic*)~}]
		\item $\sqrt{5x+1}=x+1$
		\item $x=\sqrt{3x+7}-3$
		\hfill\null
	\end{enumerate*}
\end{exercise}
	
\vfill

	\begin{center} \hfill
		\raisebox{0.4em}{
			\rotatebox{\rotationdegree}{
				\parbox{\textwidth}{
					\begin{enumerate*}[label={\theexer~(\arabic*)~}]
						\item $x=0$ or $x=3$ 
						\item $x=-2$ or $x=-1$
						\hfill\null
					\end{enumerate*}
				}
			}
		}
	\end{center}

	\newpage

	\begin{exercise}
		Solve each radical equation.

	\begin{enumerate*}[label={(\arabic*)~}]
		\item $\sqrt{6x+7}-x=2$
		\item $\sqrt{x+2}+\sqrt{x-1}=3$
		\hfill\null
	\end{enumerate*}
\end{exercise}

%%%%%%
\vfill
\begin{center} \hfill
	\raisebox{0.4em}{
		\rotatebox{\rotationdegree}{
			\parbox{\textwidth}{
				\begin{enumerate*}[label={\theexer~(\arabic*)~}]
					\item $x=-1$ or $x=3$
					\item $x=2$
					\hfill\null
				\end{enumerate*}
			}
		}
	}
\end{center}

\begin{exercise}
		Solve each radical equation.

	\begin{enumerate*}[label={(\arabic*)~}]
		\item $\sqrt{x+5}-\sqrt{x-3}=2$
		\item $3\sqrt[3]{3x-1}=6$
		\hfill\null
	\end{enumerate*}
\end{exercise}


%%%%%%
\vfill
\begin{center} \hfill
	\raisebox{0.4em}{
		\rotatebox{\rotationdegree}{
			\parbox{\textwidth}{
				\begin{enumerate*}[label={\theexer~(\arabic*)~}]
					\item $x=4$
					\item $x=3$
					\hfill\null
				\end{enumerate*}
			}
		}
	}
\end{center}

\newpage 

\begin{exercise}
	Solve each radical equation.

\begin{enumerate*}[label={(\arabic*)~}]
	\item $(x+3)^{\frac12}=x+1$
	\item $2(x-1)^{\frac12}-(x-1)^{-\frac12}=1$
	\hfill\null
\end{enumerate*}
\end{exercise}


%%%%%%
\vfill
\begin{center} \hfill
\raisebox{0.4em}{
	\rotatebox{\rotationdegree}{
		\parbox{\textwidth}{
			\begin{enumerate*}[label={\theexer~(\arabic*)~}]
				\item $x=1$
				\item $x=2$
				\hfill\null
			\end{enumerate*}
		}
	}
}
\end{center}

\begin{exercise}
	Solve each radical equation.

\begin{enumerate*}[label={(\arabic*)~}]
	\item $(x-1)^{\frac32}=8$
	\item $(x+1)^{\frac23}=4$
	\hfill\null
\end{enumerate*}
\end{exercise}


%%%%%%
\vfill
\begin{center} \hfill
\raisebox{0.4em}{
	\rotatebox{\rotationdegree}{
		\parbox{\textwidth}{
			\begin{enumerate*}[label={\theexer~(\arabic*)~}]
				\item $x=5$
				\item $x=-9$ or $x=7$
				\hfill\null
			\end{enumerate*}
		}
	}
}
\end{center}