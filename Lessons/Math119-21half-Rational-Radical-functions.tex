% !TEX root=../MA119-Main.tex

\paragraph*{The Domain of a Rational Function}
	A rational function $f$ is defined by an equation $f(x)=\frac{p(x)}{q(x)}$, where $p(x)$ and $q(x)$ are polynomials and the degree of $q(x)$ is at least one. Since the denominator cannot be zero, the domain of $f$ consists all real numbers except the numbers such that $q(x)=0$

	\begin{example}
		Find the domain of the function $f(x)=\frac{1}{x-1}$.
	\end{example}
	\begin{solution}
		Solve the equation $x-1=0$, we get $x=1$. Then the domain is $\{x\mid x\neq 1\}$. In interval notation, the domain is
		\[(-\infty, 1)\cup (1,\infty).\]
	\end{solution}

\paragraph*{The Domain of a Radical Function}
	A radical function $f$ is defined by an equation $f(x)=\sqrt[n]{r(x)}$, where $r(x)$ is an algebraic expression.  For example $f(x)=\sqrt{x+1}$. When $n$ is odd number, $r(x)$ can be any real number.  When $n$ is even, $r(x)$ has to be nonnegative, that is $r(x)\geq 0$ so that $f(x)$ is a real number.

	\begin{example}
		Find the domain of the function $f(x)=\sqrt{x+1}$.
	\end{example}
	\begin{solution}Since the index is $2$ which is even, the function has real outputs only if the radicand $x+1\geq 0$.
		Solve the inequality, we get $x\geq -1$. In interval notation, the domain is
		\[[-1,\infty).\]
	\end{solution}

\newpage

\begin{exercise}
	Find the domain of each function. Write in interval notation.\\
	\begin{enumerate*}[label={(\arabic*)~~}]
		\item $f(x)=\frac{x^2}{x-2}$
		\item $f(x)=\frac{x}{x^2-1}$
		\item $f(x)=\sqrt{2x-3}$
		\item $f(x)=\sqrt{x^2+1}$\hfill\null
	\end{enumerate*}
\end{exercise}

%%%%%%
\vfill
\begin{center} \hfill
	\raisebox{0.4em}{
		\rotatebox{\rotationdegree}{
			\parbox{\textwidth}{
				\begin{enumerate*}[label={\theexer~(\arabic*)~}]
					\item $(-\infty, 2)\cup(2, \infty)$
					\item $(-\infty, -1)\cup (-1, 1)\cup (1, \infty)$
					\item $[\frac32, \infty)$
					\item $(-\infty, \infty)$.
					\hfill\null
				\end{enumerate*}
			}
		}
	}
\end{center}


\begin{exercise}
	Find the domain of each function. Write in interval notation.\\
	\begin{enumerate*}[label={(\arabic*)~~}]
		\item $f(x)=1-\frac{2x}{x+3}$
		\item $f(x)=\frac{x-2}{x^2-4}$
		\item $f(x)=\sqrt{1-x^2}$
		\item $f(x)=-\sqrt{\frac{1}{x-5}}$\hfill\null
	\end{enumerate*}
\end{exercise}

%%%%%%
\vfill
\begin{center} \hfill
	\raisebox{0.4em}{
		\rotatebox{\rotationdegree}{
			\parbox{\textwidth}{
				\begin{enumerate*}[label={\theexer~(\arabic*)~}]
					\item $(-\infty, -3)\cup(-3, \infty)$
					\item $(-\infty, -2)\cup (-2, 2)\cup (2, \infty)$
					\item $[-1,1]$
					\item $(5, \infty)$.
					\hfill\null
				\end{enumerate*}
			}
		}
	}
\end{center}
