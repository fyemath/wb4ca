% !TEX root=../MA119-Main.tex

\begin{tcolorbox}[title={Radical expressions}]
If $b^2=a$, then we say that $b$ is a \dfn{square root} of $a$. We denote the positive square root of $a$ as $\sqrt{a}$, called the \dfn{principle square root}.

We call the function $f(x)=\sqrt{x}$ a \dfn{square root function}. The domain of the function is all real numbers $x$ such that $x\ge 0$, in interval notation, the domain is $[0,+\infty)$. 

For any real number $a$, the expression $\sqrt{a^2}$ can be simplified as
\[\sqrt{a^2}=|a|.\]

If $b^3=a$, then we say that $b$ is a \dfn{cube root} of $a$. The cube root of a real number $a$ is denoted by $\sqrt[3]{a}$.

For any real number $a$, the expression $\sqrt[3]{a^3}$ can be simplified as
\[\sqrt[3]{a^3}=a.\]

In general, if $b^n=a$, then we say that $b$ is a \dfn{$n$-th root} of $a$.  If $n$ is \textbf{even}, the \textbf{positive} $n$-th root, called the \dfn{principal $n$-th root} is denoted by $\sqrt[n]{a}$. If $n$ is odd, we use $\sqrt[n]{a}$ to denote the $n$-the root of $a$. 

In $\sqrt[n]{a}$, the symbol $\sqrt{\phantom{a}}$ is called the \dfn{radical sign}, $a$ is called  the \dfn{radicand}, and $n$ is called the \dfn{index}.

If $n$ is even, then the $n$-th root of a negative number is not a real number.

For any real number $a$, the expression $\sqrt[n]{a^n}$ can be simplified as
\begin{enumerate}
\item If $n$ is even, then $\sqrt[n]{a^n}=|a|$.
\item If $n$ is odd, then $\sqrt[n]{a^n}=a$.
\end{enumerate}
\end{tcolorbox}

\begin{exercise}
Evaluate the square roots. If the square root is not a real number, state so.
\\
\begin{enumerate*}[label={(\arabic*)~}]
\item $-\sqrt{16}$%\hspace{1in}   
\item $\sqrt{\dfrac{4}{25}}$%\hspace{1in}  
\item $\sqrt{9}+\sqrt{49}$ %\hspace{1in} 
\item $-\sqrt{-1}$\hfill\null\null
\end{enumerate*}
\end{exercise}



%%%%%%
\vfill
\begin{center} \hfill
\raisebox{0.4em}{\rotatebox{\rotationdegree}{\parbox{0.6\textwidth}{
	\begin{enumerate*}[label={\theexer~(\arabic*)~}]
	\item $-4$
	\item $\frac25$
	\item $10$ 
	\item not a real number
	\hfill\null
	%\item $(x-2)(x-3)$
	\end{enumerate*}
}
}
}
\end{center}


\begin{exercise}Find the domain and the indicated function value.
\begin{enumerate}[label={(\arabic*)~},itemsep=1in,itemindent=1ex]
\item $f(x)=\sqrt{2x+6}$, \quad $f(-1)$, \quad $f(5)$.
\item $f(x)=-\sqrt{9-3x}$, \quad $f(3)$, \quad $f(-\frac{7}{3})$.
\end{enumerate}
\end{exercise}


%%%%%%
\vfill
\begin{center} \hfill
\raisebox{0.4em}{\rotatebox{\rotationdegree}{\parbox{0.6\textwidth}{
	\begin{enumerate*}[label={\theexer~(\arabic*)~}]
	\item $x\geq -3$, $f(-1)=2$, $f(5)=4$
	\item $x\leq 3$, $f(3)=0$, $f(-\frac73)=-4$.
	\hfill\null
	%\item $(x-2)(x-3)$
	\end{enumerate*}
}
}
}
\end{center}



\newpage


\begin{exercise}Simplify each radical expression.
\begin{enumerate}[label={(\arabic*)~}, itemsep=1in, itemindent=1ex]
\item $\sqrt{(-7)^2}$
\item $\sqrt{(x+2)^2}$
\item $\sqrt{25x^2y^6}$
\item $\sqrt[3]{-27x^3}$
\item $\sqrt[4]{16x^8}$
\item $\sqrt[5]{(2x-1)^5}$
\item $\sqrt[6]{(-2)^6}$
\item $\sqrt[7]{-1}$
\end{enumerate}
\end{exercise}


%%%%%%
\vfill
\begin{center} \hfill
\raisebox{0.4em}{\rotatebox{\rotationdegree}{\parbox{\textwidth}{
	\begin{enumerate*}[label={\theexer~(\arabic*)~}]
	\item $7$
	\item $|x+2|$
	\item $|5xy^3|$
	\item $-3x$
	\item $2x^2$
	\item $2x-1$
	\item $2$
	\item $-1$
	\hfill\null
	%\item $(x-2)(x-3)$
	\end{enumerate*}
}
}
}
\end{center}


