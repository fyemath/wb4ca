% !TEX root=../MA119-Main.tex

\paragraph*{Simplifying Complex Rational Expressions}
	A \dfn{complex rational expression} is a rational expression whose denominator or numerator contains a rational expression.

	A complex rational expression is equivalent to the quotient of its numerator by its denominator. That suggests the following strategy to simplify a complex rational expression.

	\begin{trick}
		\textbf{Simplify and Change the Viewpoint.} A complex rational expression is a quotient of two rational expressions. You may rewrite it as an multiplication by flipping the denominator. However, it's better to simply the numerator and denominator or you won't see a good looking new expression.
	\end{trick}
	% \begin{enumerate}[label={\textbf{\textup{Step \arabic*.}}~}]
	% 	\item Simplify the numerator and the denominator.
	% 	\item Rewrite the expression as the numerator multiplying the reciprocal of the denominator.
	% 	\item Multiply and simplify.
	% \end{enumerate}


	\begin{example}
		Simplify \[
			\dfrac{~\dfrac{2x-1}{x^2-1}+\dfrac{x-1}{x+1}~}{~\dfrac{x+1}{x-1}-\dfrac{1}{x^2-1}~}
		\]

		\begin{enumerate}[label={\textbf{\textup{Step \arabic*.}}~}]
			\item Simplify the numerator and the denominator.
			      \[
				      \begin{split}
						  \dfrac{~\dfrac{2x-1}{x^2-1}+\dfrac{x-1}{x+1}~}{~\dfrac{x+1}{x-1}-\dfrac{1}{x^2-1}~}
						  =&\dfrac{
						      ~\dfrac{2x-1}{(x-1)(x+1)}+\dfrac{(x-1)(x-1)}{(x-1)(x+1)}~
					      }{~
						      \dfrac{(x+1)(x+1)}{(x-1)(x+1)}-\dfrac{1}{(x-1)(x+1)}~
					      }\\[5pt]
					      =&\dfrac{
						      ~\dfrac{(2x-1)+(x-1)(x-1)}{(x-1)(x+1)}~
					      }{
						      ~\dfrac{(x+1)(x+1)-1}{(x-1)(x+1)}~
					      }\\[5pt]
					      =&\dfrac{
						      ~\dfrac{(2x-1)+(x^2-2x+1)}{(x-1)(x+1)}~
					      }{
						      ~\dfrac{(x^2+2x+1)-1}{(x-1)(x+1)}~
					      }\\[5pt]
					      =&\dfrac{
						      ~\dfrac{x^2}{(x-1)(x+1)}~
					      }{
						      ~\dfrac{x^2+2x}{(x-1)(x+1)}~
					      }
				      \end{split}
			      \]
			\item Rewrite as a product.
			      \[
				      \dfrac{
					      ~\dfrac{x^2}{(x-1)(x+1)}~
				      }{
					      ~\dfrac{x^2+2x}{(x-1)(x+1)}~
				      }
				      =\dfrac{x^2}{(x-1)(x+1)}\cdot \dfrac{(x-1)(x+1)}{x^2+2x}
			      \]
			\item Multiply and simplify.
			      \[
					  \begin{aligned}
						\dfrac{x^2}{(x-1)(x+1)}\cdot \dfrac{(x-1)(x+1)}{x^2+2x}
						=&\dfrac{x\cdot x}{(x-1)(x+1)}\cdot \dfrac{(x-1)(x+1)}{x(x+2)}\\[5pt]
						=& \dfrac{\bcancel{x}x\cancel{(x-1)(x+1)}}{\bcancel{x}(x+2)\cancel{(x-1)(x+1)}}\\[5pt]
						=& \dfrac{x}{x+2}
					  \end{aligned}
			      \]
		\end{enumerate}
	\end{example}

	\begin{note}
		Another way to simplify a complex rational expression is to multiply the LCD to both the denominator and numerator and then simplify.
	\end{note}
	%\vspace{-0.75\baselineskip}

	%\begin{example}
	%Simplify \[\dfrac{~\dfrac{x+1}{x-1}+\dfrac{x-1}{x+1}~}{~\dfrac{x+1}{x-1}-\dfrac{x-1}{x+1}~}\]
	%
	%\begin{enumerate}[label={\textbf{\textup{Step \arabic*.}}~},
	% 	 topsep=-1em,
	%    itemsep=-1ex,
	%    partopsep=1ex,
	%    parsep=1ex,
	% 	 leftmargin=*,
	%	 wide,
	%	 itemindent=0em,
	%	 ]
	%\item Simplify the numerator and denominator.
	%\[\begin{split}
	%\dfrac{~\dfrac{x+1}{x-1}+\dfrac{x-1}{x+1}~}{~\dfrac{x+1}{x-1}-\dfrac{x-1}{x+1}~}=&\dfrac{
	%	~\dfrac{(x+1)(x+1)}{(x-1)(x+1)}+\dfrac{(x-1)(x-1)}{(x-1)(x+1)}~
	%	}{~
	%	\dfrac{(x+1)(x+1)}{(x-1)(x+1)}-\dfrac{(x-1)(x-1)}{(x-1)(x+1)}~
	%	}
	%=\dfrac{
	%	~\dfrac{(x+1)(x+1)+(x-1)(x-1)}{(x-1)(x+1)}~
	%	}{
	%	~\dfrac{(x+1)(x+1)-(x-1)(x-1)}{(x-1)(x+1)}~
	%	}\\[5pt]
	%=&\dfrac{
	%	~\dfrac{(x^2+2x+1)+(x^2-2x+1)}{(x-1)(x+1)}~
	%	}{
	%	~\dfrac{(x^2+2x+1)+(x^2-2x+1)}{(x-1)(x+1)}~
	%	}
	%=\dfrac{
	%	~\dfrac{2(x^2+1)}{(x-1)(x+1)}~
	%	}{
	%	~\dfrac{4x}{(x-1)(x+1)}~
	%	}
	%\end{split}
	%\]
	%\item Rewrite as a multiplication.
	%\[
	%\dfrac{
	%	~\dfrac{2(x^2+1)}{(x-1)(x+1)}~
	%	}{
	%	~\dfrac{4x}{(x-1)(x+1)}~
	%	}
	%=\dfrac{2(x^2+1)}{(x-1)(x+1)}\cdot \dfrac{(x-1)(x+1)}{4x}
	%\]
	%\item Multiply and simplify.
	%\[\dfrac{2(x^2+1)}{(x-1)(x+1)}\cdot \dfrac{(x-1)(x+1)}{4x}
	%= \dfrac{\bcancel{2}(x^2+1)\cancel{(x-1)(x+1)}}{\bcancel{2}2x\cancel{(x-1)(x+1)}}= \dfrac{x^2+1}{2x}
	%\]
	%\end{enumerate}
	%\end{example}




\newpage

%\hrule
%%%%%%%%%%%%%%%%%%%%%%%%%%%%%%%%%%%%%%%%
\begin{exercise}
	Simplify.\\
	\begin{enumerate*}[label={(\arabic*)~}]
		\item $\dfrac{~1+\dfrac{2}{x}~}{~1-\dfrac{2}{x}~}$
		\item $\dfrac{~\dfrac{1}{x^2}-1~}{~\dfrac{1}{x^2}-\dfrac{1}{x}~}$
		\hfill\null
	\end{enumerate*}
\end{exercise}

%%%%%%
\vfill
\begin{center} \hfill
	\raisebox{0.4em}{
		\rotatebox{\rotationdegree}{
			\parbox{\textwidth}{
				\begin{enumerate*}[label={\theexer~(\arabic*)~}]
					\item $\dfrac{x+2}{x-2}$
					\item ${x+1}$
					\hfill\null
				\end{enumerate*}
			}
		}
	}
\end{center}
% \vspace{-1.5\baselineskip}

% \newpage

%%%%%%%%%%%%%%%%%%%%%%%%%%%%%%%%%%%%%%%%
\begin{exercise}
	Simplify.\\
	\begin{enumerate*}[label={(\arabic*)~}]
		\item $\dfrac{~\dfrac{x^2-y^2}{y^2}~}{~\dfrac1x-\dfrac{1}{y}~}$
		\item $\dfrac{~\dfrac{2}{(x+1)^2}-\dfrac{1}{x+1}~}{~1-\dfrac{4}{(x+1)^2}~}$
		\hfill\null
	\end{enumerate*}
\end{exercise}

%%%%%%
\vfill
\begin{center} \hfill
	\raisebox{0.4em}{
		\rotatebox{\rotationdegree}{
			\parbox{\textwidth}{
				\begin{enumerate*}[label={\theexer~(\arabic*)~}]
					\item $-\dfrac{x(x+y)}{y}$
					\item $-\dfrac1{x+3}$
					\hfill\null
				\end{enumerate*}
			}
		}
	}
\end{center}


\newpage

%%%%%%%%%%%%%%%%%%%%%%%%%%%%%%%%%%%%%%%%
\begin{exercise}
	Simplify.\\
	\begin{enumerate*}[label={(\arabic*)~~}]
		\item $\dfrac{~\dfrac{5x}{x^2-x-6}~}{~\dfrac2{x+2}+\dfrac{3}{x-3}~}$
		\item $\dfrac{~\dfrac{x+1}{x-1}+\dfrac{x-1}{x+1}~}{~\dfrac{x+1}{x-1}-\dfrac{x-1}{x+1}~}$
		\hfill\null
	\end{enumerate*}
\end{exercise}

%%%%%%
\vspace{\stretch{1.5}}
\begin{center} \hfill
	\raisebox{0.4em}{
		\rotatebox{\rotationdegree}{
			\parbox{\textwidth}{
				\begin{enumerate*}[label={\theexer~(\arabic*)~}]
					\item $1$
					\item $\dfrac{x^2+1}{2x}$
					%	\item $\dfrac{7 x + 6}{(x - 3) (x + 2)}$
					\hfill\null
					%\item $(x-2)(x-3)$
				\end{enumerate*}
			}
		}
	}
\end{center}

%%%%%%%%%%%%%%%%%%%%%%%%%%%%%%%%%%%%%%%%
\begin{exercise}
	Tim and Jim refill their cars at the same gas station twice last month. Each time Tim got \$20 gas and Jim got 8 gallon. Suppose they refill their cars on same days. The price was \$2.5 per gallon the first time. The price on the second time changed. Can you find out who had the better average price?
\end{exercise}

%%%%%%
\vfill
\begin{center} \hfill
	\raisebox{0.4em}{
		\rotatebox{\rotationdegree}{
			\parbox{\textwidth}{
				\begin{enumerate*}[label={\theexer~}]
					\item Tim had the better average price.
					\hfill\null
				\end{enumerate*}
			}
		}
	}
\end{center}